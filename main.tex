
\documentclass{article}

% Packages
\usepackage[utf8]{inputenc}
\usepackage{natbib}
\usepackage{graphicx}
\usepackage{amsthm}
\usepackage{amssymb}
\usepackage{amsmath}
\usepackage{verbatim}
\usepackage{enumitem}
\usepackage{verbatim}
\usepackage{accents}
\usepackage[normalem]{ulem}



\title{Definition and properties of Belief Dynamics in agents reasoning with argumentation}
\author{Matteo Campanelli}

\begin{document}

\maketitle
\tableofcontents

% Indentation of paragraphs
% \setlength\parindent{0pt}

% Commands
\input{mycommands.tex}
\verbatimfont{\itshape\ttfamily}

% Message
%----------
%\input{message.tex}

%\input{intuition_desiderata_trust.tex}

\input{introduction/intro_main.tex}

%\input{problem_brindisi_attempt2.tex}

% Beginning period 13 stuff
\input{potential_toc.tex}



% Beginning older than period 13 stuff


%\input{top_down_examples.tex}

% Belief dynamics framework
%\input{bdf.tex}

%\input{formalizing_invariants.tex}
	
%  -- OLD -- -
% \input{problem_period11.tex}


%\input{assumptions_this_work.tex}

%XXX: to be reincluded?? 
%\input{modelbeliefdynamics.tex}

% Dikstra dovrebbe essere considerato fuori a meno che
% non si parli della roba che lui riguardo gerarchie e simili.

%\input{related_work.tex}

%XXX: to be reincluded??
%\input{agents.tex}
%\input{update_rules.tex}
%\input{rdd.tex}


\bibliographystyle{alpha}
\bibliography{references}
\end{document}


