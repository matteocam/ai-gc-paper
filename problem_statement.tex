\section{Problem Statement}

% XXX: What is a social norm in agent societies 

% XXX: you may want to talk about what agent societies are
Social norms in multi-agent systems and agent societies are beneficial for several reasons.
They improve coordination and reduce cognitive load. % CN
However, agents cannot always agree on norms offline since properties of the society at hand may not be unknown or changing over time.
Furthermore, design of rules may be computationally hard.
% -- when the interactionist approach makes sense --
These assumptions have inspired the so-called interactionist approach to norms in multi-agent systems, % CN
where society is able to dynamically converge to norms.
Different works have studied different assumptions on this [...]. % XXX more on a brief intro for interactionist approach.

% Describe scenario
In this work we will assume a slightly different scenario. We will
assume the interactionist perspective % XXX: change phrasing
from the point of view of a system designer who aims at pushing
agents towards specific conventions. We will assume that we, as sytem
designers, know the norms we prefer more.
However, we will assume that  these norms cannot
be imposed globally (e.g. because it would take too many resources)
and that there are limited resources to push rules anyway.
% XXX: give intuitions behind scenario with specific examples?
\begin{comment}
Two cases here:
- Is it possible to push norms that are equivalent from the agent's
point of view? (coordination games)
- Is it possible to push norms that are not the same from the agent's 
point of view? And if yes, which? (Think of the Prisoner's dilemma
case).
\end{comment}
