\section{Problem Statement}

% XXX: What is a social norm in agent societies 

% XXX: you may want to talk about what agent societies are
Social norms in multi-agent systems and agent societies are beneficial for several reasons.
They improve coordination and reduce cognitive load. % CN
However, agents cannot always agree on norms offline since properties of the society at hand may  be unknown or changing over time.
Furthermore, design of rules may be computationally hard.
% -- when the interactionist approach makes sense --
These assumptions have inspired the so-called interactionist approach to norms in multi-agent systems, % CN
where society is able to dynamically converge to norms.
Different works have studied different assumptions on this [...]. % XXX more on a brief intro for interactionist approach.

% -- Assumptions on agents
One assumption on agents we would focus in this paper will be that agents cannot observe their peers' past actions
and rewards. % XXX: other features?
In several scenarios this has been shown to be sufficient for agents to converge to norms. % CN % Sen + the whole Villatoro line
% XXX ...
This set of assumptions is called ``social learning'' scenario. % CN Sen
                                % % XXX


% Describe scenario
In this work we will assume agents' adaptation mechanisms as given and
that we are
within the social learning framework.
We study a slightly more specific scenario  than the traditional one.
In particular the question we want to ask is the following:

\begin{center}
\emph{
Can we achieve the  emergence of a specific conventions in a society without any
additional reward or punishment on agents and minimal deployment of resources?
}
\end{center}

% -- Intuitive scenario --
The following metaphor expresses the scenario we want to study:
% XXX... 

% -- More elaboration on how we are going to study this problem --
% Here I elaborate on the two main features: 
% - no punishment/reward
% - minimal deployment of resources


We will study the problem of conventions emergence when pushing 
from the point of view of a system designer who aims at pushing
agents towards specific conventions. We will assume that we, as sytem
designers, know the norms we prefer more.
However, we will assume that  these norms cannot
be imposed globally (e.g. because it would take too many resources)
and that there are limited resources to push rules anyway.
% XXX: give intuitions behind scenario with specific examples?

% -- Cases of games we want to treat -- XXX [might go later]
\begin{comment}
Two cases here:
- Is it possible to push norms that are equivalent from the agent's
point of view? (coordination games)
- Is it possible to push norms that are not the same from the agent's 
point of view? And if yes, which? (Think of the Prisoner's dilemma
case).
\end{comment}

% After 

% -- Assumptions on the types of agents --
If we are not using reward/punishment how are we going to be 
